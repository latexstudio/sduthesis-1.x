% \iffalse meta-comment
%
% This is file `sduthesis.dtx'.
%
% Copyright (C) 2012 -- 2014 by Liam Huang
% -----------------------------------
% This work may be distributed and/or modified under the
% conditions of the LaTeX Project Public License, either version 1.3
% of this license or (at your option) any later version.
% The latest version of this license is in
%   http://www.latex-project.org/lppl.txt
% and version 1.3 or later is part of all distributions of LaTeX
% version 2005/12/01 or later.
%
% This work has the LPPL maintenance status `maintained'.
%
% The Current Maintainer of this work is Liam Huang.
%
% \fi
%
% \iffalse
%<*driver>
\ProvidesFile{sduthesis.dtx}
%</driver>
%<class>\NeedsTeXFormat{LaTeX2e}[1999/12/01]
%<class>\ProvidesClass{sduthesis}
%<*class>
  [2015/01/01 v1.2.0a Thesis Template of Shandong University]
%</class>
%
%<*driver>
\documentclass{ltxdoc}
\EnableCrossrefs
\CodelineIndex
\RecordChanges
\usepackage[UTF8, fntef]{ctexcap}
\newcommand{\pkg}[1]{\textsf{#1}}
\usepackage{hologo}
\newcommand{\XeLaTeX}{\hologo{XeLaTeX}}
\begin{document}
  \DocInput{\jobname.dtx}
\end{document}
%</driver>
% \fi
%
% \CheckSum{0}
%
% \changes{v1.2.0a}{2014/01/01}{Reimplement in using DocStrip utility.}
% \changes{v1.0}{2013/05/12}{The first public release.}
%
% \GetFileInfo{\jobname.dtx}
%
% \DoNotIndex{\test}
%
% \title{The \textsf{\jobname} class\thanks{This Document
%   corresponds to \textsf{\jobname}~\fileversion,
%   dated \filedate.}}
% \author{Liam Huang \\ \texttt{liamhuang0205+\jobname@gmail.com}}
% \date{\filedate}
%
% \maketitle
%
% \begin{abstract}
%   Put text here.
% \end{abstract}
%
% \section{Usage}
%
% \DescribeMacro{\YOURMARCO}
%
% Put the description of |\YOURMARCO| here.
%
% \DescribeEnv{YOURENV}
%
% Put the description of |YOURENV| here.
%
% \StopEventually{\PrintIndex}
%
% \section{代码实现}
%
%    \begin{macrocode}
%<*class>
%    \end{macrocode}
% 载入 \pkg{kvoptions} 宏包,并进行相关设置。
%    \begin{macrocode}
\RequirePackage{kvoptions}
\RequirePackage{etoolbox}
\SetupKeyvalOptions{family=SDU, prefix=SDU@opt@, setkeys=\kvsetkeys}
\newcommand{\ekv}[1]{\kvsetkeys{SDU}{#1}}
%    \end{macrocode}
% 声明选项。
%    \begin{macrocode}
\DeclareBoolOption[true]{chsstyle}
\DeclareComplementaryOption{nochsstyle}{chsstyle}
\DeclareBoolOption[false]{print}
\DeclareComplementaryOption{noprint}{print}
\DeclareBoolOption[true]{double}
\DeclareComplementaryOption{single}{double}
\DeclareDefaultOption{\PassOptionsToPackage{\CurrentOption}{ctexcap}}
\LoadClass{ctexbook}
\ProcessKeyvalOptions*\relax
\ifSDU@opt@double\relax\else
  \PassOptionsToClass{a4paper, cs4size, UTF8}{ctexbook}
\fi
%    \end{macrocode}
% 载入所需宏包。
%    \begin{macrocode}
\RequirePackage{ifpdf,ifxetex}
\RequirePackage{geometry}
\RequirePackage{fancyhdr}
\RequirePackage{amsmath}
\RequirePackage{amsfonts}
\RequirePackage{amsthm}
\RequirePackage{amssymb}
\RequirePackage{amsbsy}
\RequirePackage{bm}
\RequirePackage{mathrsfs}
\RequirePackage{booktabs}
\RequirePackage{amsmath}
\RequirePackage{hyperref}
\ifxetex
  \RequirePackage{graphicx}
\else
  \ifpdf
    \RequirePackage{graphicx}
    \RequirePackage{epstopdf}
  \else
    \RequirePackage[dvipdfmx]{graphicx}
    \RequirePackage{bmpsize}
  \fi
\fi
\RequirePackage{xcolor}
\RequirePackage{makecell}
%    \end{macrocode}
% \begin{macro}{\fzbHei}
% 调用「方正大黑简体」作为封面大标题字体。
%
% 学校提供的 M\$ Word 版封皮,标题字体使用的是方正大黑简体。
% 然而,由于大多数的计算机,特别是学校内和学校周边的打印室没有安装这个字
% 体,因此在实际打印时,使用「中易黑体」代替。在使用 \XeLaTeX{} 时,模
% 板使用方正大黑简体;而在使用 pdf\LaTeX{} 或 \LaTeX{} 时,由于技术
% 原因,使用黑体代替。
%    \begin{macrocode}
\ifxetex
  \newCJKfontfamily[fzbighei]{\fzbHei}{FZDHTJW.ttf}
\else
  \newcommand{\fzbHei}{\heiti}
\fi
%    \end{macrocode}
% \end{macro}
% 行距设置。
%    \begin{macrocode}
\linespread{1.3}
%    \end{macrocode}
% 段落间距设置。
%    \begin{macrocode}
\setlength{\parskip}{0.3ex}
%    \end{macrocode}
% 页面布局及页边距设置。
%    \begin{macrocode}
\ifSDU@opt@double
  \geometry{left=3.75cm, right=1.75cm, top=3cm, bottom=3cm}
\else
  \geometry{left=2.75cm, right=2.75cm, top=3cm, bottom=3cm}
\fi
%    \end{macrocode}
% 页眉和页脚设置。
%    \begin{macrocode}
\pagestyle{fancy}
\fancyhf{}
\renewcommand{\headrule}{%
  \hrule\@height1.5pt\@width\headwidth%
  \vskip1pt%
  \hrule\@height\headrulewidth\@width\headwidth%
}
\fancyhead[C]{%
  \ifSDU@opt@chsstyle
    \chead{\zihao{5}山东大学学士学位论文}
  \else
    \chead{\small Shandong University Bachelor Thesis}
  \fi
}
\fancyfoot[OR, EL]{--~{\thepage}~--}
%    \end{macrocode}
% 数学环境特殊符号的定义。
%    \begin{macrocode}
\newcommand*{\me}{\ensuremath{\mathrm{e}}}
\newcommand*{\mi}{\ensuremath{\mathrm{i}}}
\newcommand*{\dif}{\ensuremath{\mathop{}\!\mathrm{d}}}
\DeclareMathAlphabet{\mathsfsl}{OT1}{cmss}{m}{sl}
\newcommand*{\VEC}[1]{\ensuremath{\boldsymbol{#1}}}
\newcommand*{\MATRIX}[1]{\ensuremath{\mathsfsl{#1}}}
\newcommand*{\TENSOR}[1]{\ensuremath{\mathsfsl{#1}}}
\newcommand*{\HUA}[1]{\ensuremath{\mathscr{{#1}}}}
\newcommand*{\SHUANG}[1]{\ensuremath{\mathbb{{#1}}}}
%    \end{macrocode}
%    \begin{macrocode}
\numberwithin{equation}{chapter}
%    \end{macrocode}
% 设置图档搜索路径。
%    \begin{macrocode}
\graphicspath{{figures/}{figure/}{pictures/}
      {picture/}{pic/}{pics/}{image/}{images/}}
%    \end{macrocode}
% 浮动体设置。
%    \begin{macrocode}
\numberwithin{figure}{chapter}
\numberwithin{table}{chapter}
\newcommand{\figcaption}{\def\@captype{figure}\caption}
\newcommand{\tabcaption}{\def\@captype{table}\caption}
%    \end{macrocode}
% 交叉引用设置。
% 使用颜色作为链接标识,设置颜色为蓝色。如果开启 |print| 模式,则
% 启用 |\hypersetup{hidelinks}|。
%    \begin{macrocode}
\hypersetup{colorlinks=true}
\hypersetup{linkcolor=black}
\ifSDU@opt@print
  \hypersetup{allcolors=black}
\else\relax\fi
%    \end{macrocode}
%
% \begin{macro}{\upcite}
% 定义参考文献上标引用。
% \end{macro}
%    \begin{macrocode}
\newcommand{\upcite}[1]{\textsuperscript{\cite{#1}}}
%    \end{macrocode}
% \begin{macro}{\enabstractname}
% 英文摘要标题的名字,默认是全部大写的 |ABSTRACT|,可以自行
% 用 \cs{renewcommand} 修改。
%    \begin{macrocode}
\newcommand\enabstractname{ABSTRACT}
%    \end{macrocode}
% \end{macro}
% \begin{macro}{\enkeywordsname}
% 英文关键字的名字。
%    \begin{macrocode}
\newcommand\enkeywordsname{Key words}
%    \end{macrocode}
% \end{macro}
% \begin{macro}{\enkeywords}
% 英文关键字命令。用法是 \cs{enkeywords}\marg{关键字列表}。
%    \begin{macrocode}
\newcommand\enkeywords[1]{%
  \vspace{1cm}\noindent{\bfseries\zihao{-4}\enkeywordsname: }#1}
%    \end{macrocode}
% \end{macro}
% \begin{macro}{\cnabstractname}
% 中文摘要标题的名字,默认是|摘\quad 要|,可以自行用 \cs{renewcommand} 修改。
%    \begin{macrocode}
\newcommand\cnabstractname{摘\quad 要}
%    \end{macrocode}
% \end{macro}
% \begin{macro}{\cnkeywordsname}
% 中文关键字的名字。
%    \begin{macrocode}
\newcommand\cnkeywordsname{关键词}
%    \end{macrocode}
% \end{macro}
% \begin{macro}{\cnkeywords}
% 中文关键字命令。用法是 \cs{cnkeywords}\marg{关键字列表}。
%    \begin{macrocode}
\newcommand\cnkeywords[1]{%
  \vspace{1cm}\noindent{\bfseries\zihao{-4}\cnkeywordsname: }#1}
%    \end{macrocode}
% \end{macro}
% \begin{environment}{enabstract}
% 英文摘要环境,按照学校要求,在结尾处分页。
%    \begin{macrocode}
\newenvironment{enabstract}{%
  \newpage
  \centering
  \begin{minipage}{.9\textwidth}
  \centerline{\zihao{-3}\bfseries\enabstractname}\vspace{.3cm}
  \centering
  \begin{minipage}{.85\textwidth}
  \setlength{\parindent}{1.2em}
}{
\end{minipage}
\end{minipage}
\clearpage
}
%    \end{macrocode}
% \end{environment}
% \begin{environment}{cnabstract}
% 中文摘要环境,按照学校要求,在结尾处分页。
%    \begin{macrocode}
\newenvironment{cnabstract}{%
  \newpage
  \centering
  \begin{minipage}{.9\textwidth}
  \centerline{\zihao{-3}\bfseries\cnabstractname}\vspace{.3cm}
  \centering
  \begin{minipage}[c]{.85\textwidth}
  \setlength{\parindent}{2em}
  \zihao{-4}
}{
\end{minipage}
\end{minipage}
\clearpage
}
%    \end{macrocode}
% \end{environment}
% 中文标题风格。
%    \begin{macrocode}
\ifSDU@opt@chsstyle
  \CTEXsetup[name={第,章}]{chapter}
  \CTEXsetup[number={\chinese{chapter}}]{chapter}
  \CTEXsetup[format={\centering}]{chapter}
  \CTEXsetup[nameformat={\bfseries\zihao{3}}]{chapter}
  \CTEXsetup[titleformat={\bfseries\zihao{3}}]{chapter}
  \CTEXsetup[aftername={\quad{}}]{chapter}
  \CTEXsetup[beforeskip={10pt}]{chapter}
  \CTEXsetup[afterskip={10pt}]{chapter}
  \CTEXsetup[name={$\S$\,,}]{section}
  \CTEXsetup[format={\bfseries\flushleft\zihao{4}}]{section}
  \CTEXsetup[format={\bfseries\flushleft\zihao{-4}}]{subsection}
  \CTEXsetup[format={\flushleft\zihao{-4}}]{subsubsection}
  \CTEXsetup[name={附录~}]{appendix}
  \CTEXsetup[number={\Alph{chapter}}]{appendix}
\else
  \PassOptionsToClass{nocap, noindent}{ctexbook}
\fi
%    \end{macrocode}
% 载入封面的定义。
%    \begin{macrocode}
\input{sduthesis-cover.def}
\input{sduthesis-statement.def}
%    \end{macrocode}
%    \begin{macrocode}
%</class>
%    \end{macrocode}
%    \begin{macrocode}
%<*cover>
%    \end{macrocode}
% 定义 tokens。
%    \begin{macrocode}
\newtoks\fenlei     % 中图分类号
\newtoks\DWdaihao   % 单位代号
\newtoks\miji       % 密级
\newtoks\StuNum     % 学号
\newtoks\Ctitle     % 中文标题
\newtoks\Cauthor    % 作者中文名
\newtoks\Cmajor     % 专业
\newtoks\Csuperver  % 导师
\newtoks\Cdate      % 中文日期
\newtoks\Dpart      % 学院
\newtoks\Grade      % 年级
%    \end{macrocode}
% \begin{macro}{\LeftLength}
% 姓名登记表格左列的宽度。
%    \begin{macrocode}
\newcommand{\LeftLength}{2.3cm}
%    \end{macrocode}
% \end{macro}
% \begin{macro}{\RightLength}
% 姓名登记表格右列的宽度。
%    \begin{macrocode}
\newcommand{\RightLength}{5.5cm}
%    \end{macrocode}
% \end{macro}
%    \begin{macrocode}
\newcommand{\Mcs}[1]{\makebox[\LeftLength][s]{{\zihao{3}\bfseries\kaishu{}#1}}}
\newcommand{\Mcc}[1]{\makebox[\RightLength][c]{{\zihao{-3}\songti{}#1}}}
%    \end{macrocode}
% \begin{macro}{\maketitlepage}
% 最终输出封面的命令。
%    \begin{macrocode}
\newcommand{\maketitlepage}{%
\thispagestyle{empty}
\begin{center}
~
\vskip 8mm\relax
{
  {\includegraphics[width = .7\textwidth]{SDUWords.jpg}}\\[3mm]
  {\scalebox{4}{\fzbHei{}毕业论文(设计)}}
}
\par \vskip 15mm \relax
{
  \begin{flushleft}
    {\zihao{3}\heiti{}论文(设计)题目:}
  \end{flushleft}
}
\vfill
{
  \noindent
  \zihao{-1}\kaishu\the\Ctitle
}
\vfill
{
  \begin{tabular}{p{\LeftLength}p{\RightLength}}
  \Mcs{姓名}& \Mcc{\the\Cauthor}\\[-.8mm]\cline{2-2}\\[-4mm]
  \Mcs{学号}& \Mcc{\the\StuNum}\\[-.8mm]\cline{2-2}\\[-4mm]
  \Mcs{学院}& \Mcc{\the\Dpart}\\[-.8mm]\cline{2-2}\\[-4mm]
  \Mcs{专业}& \Mcc{\the\Cmajor}\\[-.8mm]\cline{2-2}\\[-4mm]
  \Mcs{年级}& \Mcc{\the\Grade}\\[-.8mm]\cline{2-2}\\[-4mm]
  \Mcs{指导老师}& \Mcc{\the\Csuperver}\\[-.8mm]\cline{2-2}
  \end{tabular}
}
\par \vskip 20mm \relax
{
\zihao{3}\the\Cdate
}
\end{center}
\clearpage
}
%    \end{macrocode}
% \end{macro}
%    \begin{macrocode}
%</cover>
%    \end{macrocode}
%    \begin{macrocode}
%<*statement>
%    \end{macrocode}
% \begin{macro}{\makestatement}
% 输出诚信承诺书的命令。
%    \begin{macrocode}
\newcommand{\makestatement}{%
\begin{titlepage}
\vspace{2cm} {\zihao{4}\baselineskip=30pt

\centerline{\zihao{3}\bfseries 原\quad 创\quad 性\quad 声\quad 明}

\noindent\hspace*{2em}本人郑重声明:所呈交的学位论文,是本人在导师指导下,独
立进行研究所取得的成果。除文中已经注明引用的内容外,本论文不包
含任何其他个人或集体已经发表或撰写过的科研成果。对本论文的研究作出重
要贡献的个人和集体,均已在文中以明确方式标明。本声明
的法律责任由本人承担。

\vspace{13mm}

\noindent\hspace*{2em}论文作者签名:\hrulefill \hspace{1em}日\hspace{1em} 期:\hrulefill

\vspace{2.7cm}

\centerline{\zihao{3}\bfseries 关于学位论文使用授权的声明}

\noindent\hspace*{2em}本人完全了解山东大学有关保留、使用学位论文的规定,同意学
校保留或向国家有关部门或机构送交论文的复印件和电子版,允许论文被查阅
和借阅;本人授权山东大学可以将本学位论文全部或部分内容编入有关数据库
进行检索,可以采用影印、缩印或其他复制手段保存论文和汇编本学位论文。

\noindent\hspace*{2em}(保密的论文在解密后应遵守此规定)

\vspace{13mm}

\noindent\hspace*{2em}论文作者签名:\hrulefill\hrulefill\hspace{0.5em} 导师签名:\hrulefill\hrulefill\hspace{0.5em}日\hspace{0.5em}期:\hrulefill \hrulefill }
\end{titlepage}
}
%    \end{macrocode}
% \end{macro}
%    \begin{macrocode}
%</statement>
%    \end{macrocode}
%    \begin{macrocode}
%<*class>
%    \end{macrocode}
% \begin{macro}{\maketitlepagestatement}
% 输出标题页和承诺书,并设置 \cs{frontmatter}。
%    \begin{macrocode}
\newcommand{\maketitlepagestatement}{%
\maketitlepage
\thispagestyle{empty}
~
\vfill\eject
\thispagestyle{empty}
\makestatement
\vfill\eject
\thispagestyle{empty}
~
\vfill\eject
\setcounter{page}{1}
\frontmatter
}
%    \end{macrocode}
% \end{macro}
% \begin{macro}{\tableofcontents}
% 重定义目录命令。
%    \begin{macrocode}
\let\savedtableofcontents\tableofcontents
\renewcommand{\tableofcontents}{%
  \savedtableofcontents
    \ifSDU@opt@double
    \cleardoublepage
  \else
    \clearpage
  \fi
  \mainmatter
}
%    \end{macrocode}
% \end{macro}
%    \begin{macrocode}
%</class>
%<class>\endinput
%    \end{macrocode}
% \iffalse
%<*readme>
# The `sduthesis` Class / `sduthesis` 文档类

## Introduction / 介绍

The `sduthesis` is designed for students of Shandong Univ., P.R.China,
by [Liam Huang][liam-ctan]. The 1.0.x versions of `sduthesis` were
released in the name of Ch'en Meng, while from the begining of version 1.2.0,
it was released in the name of Liam Huang and was rewritten in `docstrip`.

This work is released under the LaTeX Project Public License, v1.3c or later.
See the License file.

`sduthesis` 是由 [Liam Huang][liam-ctan] 为山东大学学生设计的 LaTeX 论文模板。
1.x 以化名 Ch'en Meng 的名义发布;1.2.0 版本开始,以 `docstrip` 工具重写了整个代码,
并以 Liam Huang 的名义发布。

`sduthesis` 遵循不低于 1.3 版本的 LPPL 许可证,详情请查看 LICENSE 文件。

## Author / 作者

Liam Huang

Email: liamhuang0205+sduthesis@gmail.com

If you are interested in the process of development you may observe

<https://github.com/LiamHuang0205/sduthesis>

[liam-ctan]: http://www.ctan.org/author/huang-l
%</readme>
%<*license>
Released under the [LaTeX Project Public
License](http://www.latex-project.org/lppl.txt), v1.3c or later.

The package has status 'maintained': the current maintainer is
[Liam Huang](liamhuang0205+sduthesis@gmail.com).
%</license>
%\fi
% \Finale
\endinput
